\documentclass{article}
\usepackage[utf8]{inputenc}
\usepackage[a4paper,top=2cm,bottom=2cm,left=3cm,right=3cm,marginparwidth=1.75cm]{geometry}
\geometry{a4paper} 
\usepackage{amsmath}
\usepackage{physics}
\usepackage{color,soul}
\DeclareMathOperator{\di}{d\!}
\newcommand*\Eval[3]{\left.#1\right\rvert_{#2}^{#3}}
\newcommand{\inlist}[1]{\texttt{#1}}

\author{J L Kaplan}

\begin{document}

\section*{SICP Exercise 1.19}

\subsection*{Derivation of $p'$ and $q'$}
First transformation:
\begin{equation}
  a_{n+1} \leftarrow q \left( a_n + b_n \right) + p a_n, \qquad b_{n+1} \leftarrow p b_n + q a_n,
\end{equation}
second transformation:
\begin{equation}
  \label{eq:t2}
  a_{n+2} \leftarrow q \left( a_{n+1} + b_{n+1} \right) + p a_{n+1}, \qquad b_{n+2} \leftarrow p b_{n+1} + q a_{n+1},
\end{equation}
then substituting the values from the first transformation into the $b_{n+2}$ transformation in equation~\ref{eq:t2}, we can see that:
\begin{equation}
  b_{n+2} \leftarrow b_n \left( p^2 + q^2 \right) + a_n \left( 2pq + q^2 \right),
\end{equation}
and comparing with transform 1 it can be seen that
\begin{equation}
  p' = p^2 + q^2,
\end{equation}
and
\begin{equation}
  q' = 2pq + q^2.
\end{equation}


\end{document}
